% !TEX root = ./thesis.tex

\chapter{Minimally perturbed luciferins for multicomponent imaging and cross coupling}
\section{Introduction}
% REVIEW place all the stuff below under the Cycluc heading?
With a method for selection of orthogonal luciferin-luciferase pairs in hand, we turned to optimization of these pairings for \invivo{} use. Our kinetic studies of initial pairings indicated that our sterically modified luciferins had drastically increased K\textsubscript{m} values (\autoref{tb:kinetics}). Not only was this value inflated for the negative pairing, it was also high in the positively-paired mutant-analog combo. Previous work in collaboration with the Miller lab showed that a reduction in luciferin K\textsubscript{m} resulted in improvements in \invivo{} imaging.\cite{RN98}
Conveniently, the cyclic luciferin (CycLuc) scaffold developed in the Miller lab contained modifications to only the 5' and 6' positions of the luciferin scaffold, leaving the 4' and 7' positions unmodified.\cite{RN101,Reddy:2010gaa}
We hypothesized that addition of our lead modifications to these cites of the CycLuc scaffold may decrease K\textsubscript{m}, yet retain selectivity in our lead pairings.
As demonstrated in \autoref{chap:orthog} and \autoref{chap:parallel}, the most promising compounds in terms of orthogonality and brightness were the methyl and bromo series (in addition to \dluc).
Making such compounds necessitated a departure from the synthetic route demonstrated by Miller,\cite{Reddy:2010gaa} because the necessary indoline starting materials were not available.
% TODO figure here with the proposed retrosynth alongside Miller? Should be in a presentation.

\section{Brominated CycLuc luciferins for bright, multicomponent imaging}

\section{Suzuki cross-coupling of halogenated luciferins}

\section{Conclusions and future directions}

\bibliographystyle{achemso}
\bibliography{thesis2}

%%% Local Variables: ***
%%% mode: latex ***
%%% TeX-master: "thesis.tex" ***
%%% End: ***
