% !TEX root = ./thesis.tex

\chapter{Minimally perturbed luciferins for multicomponent imaging and cross coupling}
\section{Introduction}


\section{Brominated CycLuc luciferins for bright, multicomponent imaging}

%%%% Figure: Retrosynthesis
\begin{figure}[htbp]
\includegraphics[scale=0.8]{cross_coupling_luciferins/cyclucRetro} % use scale for chemdraws only
\centering
\caption[Cyclic luciferin retrosyntheses.]{
Cyclic luciferin retrosyntheses. \textit{Top:} Miller's retrosynthesis of CycLuc1. \textit{Bottom:} Proposed retrosynthesis of modified CycLuc scaffolds using our Appel's salt chemistry.
}
  \label{fig:cyclucRetro}
\end{figure}
%%%%
With a method for selection of orthogonal luciferin-luciferase pairs in hand, we turned to optimization of these pairings for \invivo{} use. Our kinetic studies of initial pairings indicated that our sterically modified luciferins had drastically increased K\textsubscript{m} values (\autoref{tb:kinetics}). Not only was this value inflated for the negative pairing, it was also high in the positively-paired mutant-analog combo. Previous work in collaboration with the Miller lab showed that a reduction in luciferin K\textsubscript{m} resulted in improvements in \invivo{} imaging.\cite{Evanssyntheticluciferinimproves2014}
Conveniently, the cyclic luciferin (CycLuc) scaffold developed in the Miller lab contained modifications to only the 5' and 6' positions of the luciferin scaffold, leaving the 4' and 7' positions unmodified.\cite{ReddyRobustlightemission2010,MoffordAminoluciferinsExtendFirefly2014}
We hypothesized that addition of our lead modifications to these cites of the CycLuc scaffold may decrease K\textsubscript{m}, yet retain selectivity in our lead pairings.
As demonstrated in \autoref{chap:orthog} and \autoref{chap:parallel}, the most promising compounds in terms of orthogonality and brightness were the methyl and bromo series (in addition to \dluc).
We also thought that we could produce such compounds via only slight modification to our previously reported route to other brominated luciferins (\autoref{fig:cyclucRetro}).\cite{SteinhardtBrominatedLuciferinsAre2016}
Appel's salt would be used to install the fuzed thiazole ring to the indoline, while simultaneously installing the cyano group for final condensation with cysteine to produce the luciferin.
I initially targeted the 4' and 7' bromoluciferins, as these were compounds that performed well in our script analysis, and were amenable to further diversification via cross coupling.
\begin{thoughts}
  Synthesis of these compounds were successful via use of a TFA protecting group on the indoline nitrogen. Turns out that TFA is the overall best protecting group, offering a good combination of installation and removal, and electronics for the ring. I also tried Boc and methyl (to make the CycLuc2 variant), and those were either too electron rich, or too hard to remove.
\end{thoughts}
Indoline bearing a nitro group at the necessary position is commercially available, and is easily TFA protected (\autoref{fig:4BrCycLuc}).\cite{ReddyRobustlightemission2010,MoffordAminoluciferinsExtendFirefly2014}
Reduction with palladium on carbon proceeded smoothly, and without need for purification.
The indoline was then dibrominated, and condensed with Appel's salt, via our previously described sequence.\cite{SteinhardtBrominatedLuciferinsAre2016}
Palladium mediated cyclization of this thioformamide gave a mixture of two regioisomers, which were separable by column chromatography.
Taking the desired isomer forward, a subsequent one-pot depretection and condensation with \Dcys{} gave \textbf{4'--BrCycLuc}.
\begin{thoughts}
  The undesired isomer was also taken forward and condensed. I called it \textbf{DaveLuc} in memory of all the Daves that we have had in lab. \textbf{DaveLuc} performed well in initial dose-responses, and might be an interesting 4'7' disubstituted compound to look at in the future.
\end{thoughts}

%%%% Figure: 4'BrCycLuc synthesis
\begin{figure}[htbp]
\includegraphics[scale=0.8]{cross_coupling_luciferins/4BrCycluc} % use scale for chemdraws only
\centering
\caption[Synthesis of \textbf{4'--BrCycLuc.}]{
Synthesis of \textbf{4'--BrCycLuc.} Our route, including Appel's salt condensation proved general for the CycLuc scaffolds.
}
  \label{fig:4BrCycLuc}
\end{figure}
%%%%

Synthesis of \textbf{7'--BrCycLuc} proceeded similarly, with bromination at the 7' position occurring later in the sequence following deprotection of the benzothiazole.
\begin{thoughts}
  In this case, the Boc protecting group was used for historical reasons (\bf{7'-BrCycLuc} was actually synthesized first) if this is ever made again, I would recommend starting with the TFA protecting group (see \autoref{fig:4BrCycLuc}).
\end{thoughts}

%%%% Figure: 7'BrCycLuc synthesis
\begin{figure}[htbp]
\includegraphics[scale=0.8]{cross_coupling_luciferins/7BrCycluc} % use scale for chemdraws only
\centering
\caption[Synthesis of \textbf{7'--BrCycLuc.}]{
Synthesis of \textbf{7'--BrCycLuc.} Our route including Appel's salt condensation proved general for the CycLuc scaffolds.
}
  \label{fig:7BrCycLuc}
\end{figure}
%%%%

Once synthesized, these compounds were evaluated with purified wild-type luciferase in a dose-response assay. Due to the CycLuc core, we expected a decrease in apparent K\textsubscript{M} relative to \bf{4'-} and \bf{7'-BrLuc}. 

\section{Suzuki cross-coupling of halogenated luciferins}

\section{Conclusions and future directions}

\bibliographystyle{achemso}
\bibliography{bibs/zoteroLib}

%%% Local Variables: ***
%%% mode: latex ***
%%% TeX-master: "thesis.tex" ***
%%% End: ***
