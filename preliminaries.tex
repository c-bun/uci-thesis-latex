\thesistitle{Parallel screening for rapid identification of orthogonal bioluminescent tools}

\degreename{Doctor of Philosophy}

% Use the wording given in the official list of degrees awarded by UCI:
% http://www.rgs.uci.edu/grad/academic/degrees_offered.htm
\degreefield{Organic Chemistry}

% Your name as it appears on official UCI records.
\authorname{Colin Michael Rathbun}

% Use the full name of each committee member.
\committeechair{Associate Professor Jennifer Prescher}
\othercommitteemembers
{
  Professor Scott Rychnovsky\\
  Professor Chris Vanderwal
}

\degreeyear{2012}

\copyrightdeclaration
{
  {\copyright} {\Degreeyear} \Authorname
}

% If you have previously published parts of your manuscript, you must list the
% copyright holders; see Section 3.2 of the UCI Thesis and Dissertation Manual.
% Otherwise, this section may be omitted.
% \prepublishedcopyrightdeclaration
% {
% 	Chapter 4 {\copyright} 2003 Springer-Verlag \\
% 	Portion of Chapter 5 {\copyright} 1999 John Wiley \& Sons, Inc. \\
% 	All other materials {\copyright} {\Degreeyear} \Authorname
% }

% The dedication page is optional.
\dedications
{
  (Optional dedication page)
  xxx
  To ...
}

\acknowledgments
{
  I would like to thank the National Science Foundation for funding through the Graduate Research Fellowship Program (grant No. DGE-1321846, and Allergan for funding through the Allergan Graduate Fellowship. I would also like to thank the members of the Prescher lab, Bernard Choi, and Bruce Tromberg, for very helpful discussions. xxx

  You also need to acknowledge any publishers of your previous
  work who have given you permission to incorporate that work
  into your dissertation. See Section 3.2 of the UCI Thesis and
  Dissertation Manual.)
}


% Some custom commands for your list of publications and software.
\newcommand{\mypubentry}[3]{
  \begin{tabular*}{1\textwidth}{@{\extracolsep{\fill}}p{4.5in}r}
    \textbf{#1} & \textbf{#2} \\
    \multicolumn{2}{@{\extracolsep{\fill}}p{.95\textwidth}}{#3}\vspace{6pt} \\
  \end{tabular*}
}
\newcommand{\mysoftentry}[3]{
  \begin{tabular*}{1\textwidth}{@{\extracolsep{\fill}}lr}
    \textbf{#1} & \url{#2} \\
    \multicolumn{2}{@{\extracolsep{\fill}}p{.95\textwidth}}
    {\emph{#3}}\vspace{-6pt} \\
  \end{tabular*}
}

% Include, at minimum, a listing of your degrees and educational
% achievements with dates and the school where the degrees were
% earned. This should include the degree currently being
% attained. Other than that it's mostly up to you what to include here
% and how to format it, below is just an example.
\curriculumvitae
{

\textbf{EDUCATION}

  \begin{tabular*}{1\textwidth}{@{\extracolsep{\fill}}lr}
    \textbf{Doctor of Philosophy in Computer Science} & \textbf{2012} \\
    \vspace{6pt}
    University name & \emph{City, State} \\
    \textbf{Bachelor of Science in Computational Sciences} & \textbf{2007} \\
    \vspace{6pt}
    Another university name & \emph{City, State} \\
  \end{tabular*}

\vspace{12pt}
\textbf{RESEARCH EXPERIENCE}

  \begin{tabular*}{1\textwidth}{@{\extracolsep{\fill}}lr}
    \textbf{Graduate Research Assistant} & \textbf{2007--2012} \\
    \vspace{6pt}
    University of California, Irvine & \emph{Irvine, California} \\
  \end{tabular*}

\vspace{12pt}
\textbf{TEACHING EXPERIENCE}

  \begin{tabular*}{1\textwidth}{@{\extracolsep{\fill}}lr}
    \textbf{Teaching Assistant} & \textbf{2009--2010} \\
    \vspace{6pt}
    University name & \emph{City, State} \\
  \end{tabular*}

\pagebreak

\textbf{REFEREED JOURNAL PUBLICATIONS}

  \mypubentry{Ground-breaking article}{2012}{Journal name}

\vspace{12pt}
\textbf{REFEREED CONFERENCE PUBLICATIONS}

  \mypubentry{Awesome paper}{Jun 2011}{Conference name}
  \mypubentry{Another awesome paper}{Aug 2012}{Conference name}

\vspace{12pt}
\textbf{SOFTWARE}

  \mysoftentry{Magical tool}{http://your.url.here/}
  {C++ algorithm that solves TSP in polynomial time.}

}

% The abstract should not be over 350 words, although that's
% supposedly somewhat of a soft constraint.
\thesisabstract
{
Genetically-encoded fluorescent probes have revolutionized our understanding of biological systems. However, the transition of fluorescent probes in vivo has been hampered by the opacity of tissue and its propensity for autofluorescence. A complementary imaging technology, bioluminescence, does not suffer from these complications because it does not require excitation light. Thus, the technique is exquisitely sensitive—-with the ability to see as few as ten cells in a mouse. Bioluminescence relies on luciferase enzymes that catalyze the oxidation of small-molecule substrates (luciferins), releasing photons of light in the process. Unfortunately, the optimal luciferases for in vivo use rely on the same luciferins, precluding studies of more than one feature at a time.

To address this issue, I developed and analyzed a number of new luciferin probes, and created a selection platform to find mutually orthogonal luciferases and luciferins for multicomponent imaging. In contrast to the spectroscopic resolution of fluorescent tools, these probes were designed to exhibit substrate resolution. Combining luciferin analogs and mutant enzymes, we tested 20 luciferins with 207 luciferases, generating 4,140 enzyme-substrate combinations, and thus a potential for more than 4 million possible sets. Since it would be impractical to evaluate these manually, I derived a mathematical quantification of orthogonality to score each potential pairing. Next, I wrote a supercomputer algorithm to search this dataset for the highest-scoring pairs. The software provided a ranked list of mutually orthogonal enzyme-substrate pairs that were biochemically verified. Resolution was maintained when these probes were moved into mouse models, highlighting the speed and accuracy of my approach.

My most recent work focused on increasing the practicality of these tools for preclinical imaging. The major drawbacks of our approach included temporal resolution and background emission. I addressed these issues by utilizing traditional spectral unmixing algorithms to deconvolute substrate signals mathematically. This enabled sequential imaging of substrates, and the ability to resolve smaller numbers of cells. With a top orthogonal pair I ``unmixed'' gradients of mutant luciferases in bacterial lysate and resolved mixtures of these mutants in mouse tumor models. I am currently applying these techniques to track cells involved in metastasis in mouse models.
}


%%% Local Variables: ***
%%% mode: latex ***
%%% TeX-master: "thesis.tex" ***
%%% End: ***
