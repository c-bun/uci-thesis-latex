% !TEX root = ./thesis.tex

\chapter{Brominated Luciferins Are Versatile Bioluminescent
Probes}
\section{Introduction}
Recent years have seen a surge of interest in accessing novel luciferins
for bioluminescence imaging
(BLI).{\protect\hyperlink{cbic201600564-bib-0001}{1}-\protect\hyperlink{cbic201600564-bib-0005}{5}}
BLI relies on light generation by luciferase enzymes and luciferin small
molecules.{\protect\hyperlink{cbic201600564-bib-0006}{6}} These probes
are routinely used in vitro and in vivo for monitoring diverse
biological processes.{\protect\hyperlink{cbic201600564-bib-0007}{7}}
However, a limited supply of robust, light‐emitting scaffolds has
stymied efforts to visualize cellular communication and other
multicomponent processes in vivo. We and others are attempting to fill
this void with new luciferin
architectures.{\protect\hyperlink{cbic201600564-bib-0007}{7},
\protect\hyperlink{cbic201600564-bib-0008}{8}} To date, these efforts
have focused on analogues of {d}‐luciferin, the native light‐emitting
substrate for firefly luciferase (Fluc,
Figure \protect\hyperlink{cbic201600564-fig-0001}{1} A).{\protect\hyperlink{cbic201600564-bib-0001}{1},
\protect\hyperlink{cbic201600564-bib-0004}{4},
\protect\hyperlink{cbic201600564-bib-0006}{6}} Fluc can tolerate a
variety of modified substrates, although most are not processed as
efficiently as {d}‐luciferin, resulting in reduced photon
production.{\protect\hyperlink{cbic201600564-bib-0004}{4},
\protect\hyperlink{cbic201600564-bib-0005}{5},
\protect\hyperlink{cbic201600564-bib-0009}{9}} Improved light outputs
have been achieved, in some cases, by using mutant
luciferases.{\protect\hyperlink{cbic201600564-bib-0010}{10},
\protect\hyperlink{cbic201600564-bib-0011}{11}}

% TODO overview figure
\begin{figure}
\centering
\includegraphics{//wol-prod-cdn.literatumonline.com/cms/attachment/39d2313d-cf86-498f-99e7-cd3de4fa5bd8/cbic201600564-fig-0001-m.jpg}
\caption{\textbf{Figure 1} ¶ \protect\hyperlink{}{Open in figure
viewer}\href{/action/downloadFigures?id=cbic201600564-fig-0001\&doi=10.1002\%2Fcbic.201600564}{PowerPoint}
¶ Bioluminescence imaging with luciferases and luciferins. A) Fluc
catalyzes the oxidation of {d}‐luciferin. The reaction proceeds through
an excited‐state (S\textsubscript{1}) oxyluciferin intermediate.
Relaxation to the ground state (S\textsubscript{0}) results in photon
release; a) firefly luciferase, Mg\textsuperscript{2+}, ATP. B) Crystal
structure of a luciferin adenylate mimic bound to Fluc (PDB ID: 4G36).
Targeted positions on the luciferin core are highlighted. C) Brominated
analogues investigated in this study. ¶
\protect\hyperlink{}{{Caption}\emph{}} ¶ Bioluminescence imaging with
luciferases and luciferins. A) Fluc catalyzes the oxidation of
d‐luciferin. The reaction proceeds through an excited‐state (S1)
oxyluciferin intermediate. Relaxation to the ground state (S0) results
in photon release; a) firefly luciferase, Mg2+, ATP. B) Crystal
structure of a luciferin adenylate mimic bound to Fluc (PDB ID: 4G36).
Targeted positions on the luciferin core are highlighted. C) Brominated
analogues investigated in this study.}
\end{figure}

Efforts to develop new bioluminescent tools would benefit from a broad
assortment of bright luciferins. Historically, such scaffolds have been
difficult to access, owing to their densely functionalized cores.
Late‐stage modifications to {d}‐luciferin are also non‐trivial,
preventing the rapid production of new analogues. To address these
issues, we aimed to prepare the bromo‐substituted scaffolds (4′‐BrLuc,
5′‐BrLuc, and 7′‐BrLuc) shown in
Figure \protect\hyperlink{cbic201600564-fig-0001}{1}. Bromo groups would
impart a modest steric perturbation to the luciferin core but could
serve as versatile chemical handles for downstream cross‐coupling
reactions and novel probe
development.{\protect\hyperlink{cbic201600564-bib-0012}{12}-\protect\hyperlink{cbic201600564-bib-0014}{14}}
The luciferins would also likely be tolerated by Fluc or related
mutants. We envisioned that the C5′ bromo appendage would fit into the
active site, as a relatively large enzyme pocket is found in this
position
(Figure \protect\hyperlink{cbic201600564-fig-0001}{1} B).{\protect\hyperlink{cbic201600564-bib-0015}{15}}
Fluc can also process many 5′‐substituted
luciferins.{\protect\hyperlink{cbic201600564-bib-0016}{16}-\protect\hyperlink{cbic201600564-bib-0019}{19}}
Bromo groups at C4′ and C7′, by contrast, could present a steric barrier
to Fluc utilization, as these positions lie in close proximity to the
enzyme backbone. Such substrates would be excellent candidates for
engineered (orthogonal) luciferases, though, provided that they are
capable of robust light emission.

Our selection of bromo substituents was predicated on the functional
groups not interfering with photon production. To assess this parameter,
we employed time‐dependent density functional
theory{\protect\hyperlink{cbic201600564-bib-0020}{20}} (TDDFT) to
analyze the adiabatic emissions of relevant oxyluciferins from the
excited state (S\textsubscript{1}) to the ground state
(S\textsubscript{0}). The performance of TDDFT for computing the
intensities of such emissions is well established: the oscillator
strength is proportional to the spontaneous emission rate from
S\textsubscript{1} and thus tracks with light
output.{\protect\hyperlink{cbic201600564-bib-0021}{21},
\protect\hyperlink{cbic201600564-bib-0022}{22}} When we calculated
oscillator strengths for a variety of known luciferins, we found that
the values correlated with reported bioluminescent outputs
(Table \protect\hyperlink{cbic201600564-tbl-0001}{1}). For example,
{d}‐luciferin analogues lacking electron‐donating groups at the 6′
position (and thus poor emitters) were predicted to have low oscillator
strengths.{\protect\hyperlink{cbic201600564-bib-0023}{23}-\protect\hyperlink{cbic201600564-bib-0027}{27}}
By contrast, analogues with 6′‐amino substituents---and known light
emitters---were predicted to have oscillator strengths on par with the
native substrate. TDDFT analyses performed on 4′‐BrLuc, 5′‐BrLuc, and
7′‐BrLuc suggested that the bromo substituents would minimally impact
bioluminescent output
(Table \protect\hyperlink{cbic201600564-tbl-0001}{1}).

% TODO oscillator stren table
{Table 1. }Comparison of calculated oscillator strengths and
bioluminescence emission intensities.
\begin{itemize}
\tightlist
\item
  \protect\hypertarget{cbic201600564-tnote-0001}{}{{[}a{]} Calculated as
  a theoretical maximum. {[}b{]} Bioluminescence was measured by using
  100 μ{m} luciferin and 1 μg Fluc. {[}c{]} No signal observed.}
\end{itemize}

Encouraged by these results, we set out to synthesize the target
brominated luciferins. The desired scaffolds all comprised benzothiazole
cores; such structures are readily accessible from anilines and Appel's
salt chemistry
(Scheme \protect\hyperlink{cbic201600564-fig-5001}{2}).{\protect\hyperlink{cbic201600564-bib-0028}{28}}
En route to 4′‐BrLuc, the dibrominated aniline \textbf{1} was first
condensed with Appel's salt. The resulting dithiazole (\textbf{2}) was
then fragmented and cyclized to provide cyanobenzothiazole \textbf{3}
(Scheme \protect\hyperlink{cbic201600564-fig-5001}{2} A). The
benzothiazole precursors to 5′‐BrLuc and 7′‐BrLuc were prepared
analogously (Scheme \protect\hyperlink{cbic201600564-fig-5001}{2} B--C).
In the latter case, the bromine substituent was installed
post‐cyclization. The desired luciferins 4′‐BrLuc, 5′‐BrLuc, and
7′‐BrLuc were ultimately accessed through {d}‐cysteine condensations
with the appropriate cyanobenzothiazoles.

% TODO synthesis scheme
\begin{figure}
\centering
\includegraphics{//wol-prod-cdn.literatumonline.com/cms/attachment/2b2cf202-e730-464b-af2a-4862c1bb656f/cbic201600564-fig-5001-m.jpg}
\caption{\textbf{Scheme 1} ¶ \protect\hyperlink{}{Open in figure
viewer}\href{/action/downloadFigures?id=cbic201600564-fig-5001\&doi=10.1002\%2Fcbic.201600564}{PowerPoint}
¶ Synthesis of brominated luciferin analogues.
a) BTMABr\textsubscript{3} (2.2 equiv); b) pyridine (2.2 equiv); c) DBU
(2.9 equiv); d) PdCl\textsubscript{2} (0.1 equiv), CuI (0.5 equiv), TBAB
(2.1 equiv); e) BCl\textsubscript{3} (6.0 equiv); f) {d}‐Cys
(1.2 equiv); g) NBS (1.5 equiv). ¶
\protect\hyperlink{}{{Caption}\emph{}} ¶ Synthesis of brominated
luciferin analogues. a) BTMABr3 (2.2 equiv); b) pyridine (2.2 equiv);
c) DBU (2.9 equiv); d) PdCl2 (0.1 equiv), CuI (0.5 equiv), TBAB
(2.1 equiv); e) BCl3 (6.0 equiv); f) d‐Cys (1.2 equiv); g) NBS
(1.5 equiv).}
\end{figure}

With the analogues in hand, we analyzed their light‐emitting properties.
The compounds were incubated with recombinant Fluc, and bioluminescent
photon production was measured. As shown in
Figure \protect\hyperlink{cbic201600564-fig-0002}{3}, all analogues
produced light in a dose‐dependent fashion. Compared to the native
substrate, only 5′‐BrLuc exhibited robust levels of emission. As noted
earlier, steric clashes might have prevented efficient processing of
4′‐BrLuc and 7′‐BrLuc. Interestingly, the apparent
\emph{K}\textsubscript{m} value for 4′‐BrLuc was on par with that for
{d}‐Luc, but the relative \emph{k}\textsubscript{cat} value was reduced
50‐fold (Table \protect\hyperlink{cbic201600564-tbl-0002}{2}). These
results indicated that the halogen atom at this position might have room
to dock but might interfere with catalysis. The enzymatic parameters for
5′‐BrLuc, as expected, were similar to those of the native substrate.
Bioluminescence spectra for the brominated luciferins were also
red‐shifted from {d}‐Luc, whereas the fluorescence spectra were
virtually identical (Figure S1 in the Supporting Information). These
results indicated that the analogues could access alternate excited
state geometries and relaxation pathways in the Fluc active site.

% TODO WT light emission
\begin{figure}
\centering
\includegraphics{//wol-prod-cdn.literatumonline.com/cms/attachment/548c4ac6-5aeb-4a77-b14b-736f51a925bf/cbic201600564-fig-0002-m.jpg}
\caption{\textbf{Figure 2} ¶ \protect\hyperlink{}{Open in figure
viewer}\href{/action/downloadFigures?id=cbic201600564-fig-0002\&doi=10.1002\%2Fcbic.201600564}{PowerPoint}
¶ Differential bioluminescent photon production is observed with bromo
luciferins and recombinant Fluc. Sample bioluminescence images are
shown. Error bars represent the standard deviation of the mean for
\emph{n}=3 experiments. Signal at 10\textsuperscript{5} photons per s
represents background luminescence. ¶
\protect\hyperlink{}{{Caption}\emph{}} ¶ Differential bioluminescent
photon production is observed with bromo luciferins and recombinant
Fluc. Sample bioluminescence images are shown. Error bars represent the
standard deviation of the mean for n=3 experiments. Signal at 105
photons per s represents background luminescence.}
\end{figure}

% TODO enzymatic parameters
\hypertarget{cbic201600564-tbl-0002}{}
{Table 2. }Enzymatic parameters for Fluc‐catalyzed light emission.
\begin{itemize}
\tightlist
\item
  \protect\hypertarget{cbic201600564-tnote-0002}{}{{[}a{]} Expressed as
  percent relative to {d}‐Luc.}
\end{itemize}

Although photon measurements suggested that 5′‐BrLuc was the best Fluc
substrate, it is possible that the 4′ and 7′ isomers were simply
inherently weaker emitters (as suggested by TDDFT calculations,
Table \protect\hyperlink{cbic201600564-tbl-0001}{1}). To explore this
possibility, we used a traditional chemiluminescence assay to measure
each analogue's intrinisic ability to produce light
(Figure \protect\hyperlink{cbic201600564-fig-0003}{4}).{\protect\hyperlink{cbic201600564-bib-0029}{29}-\protect\hyperlink{cbic201600564-bib-0031}{31}}
This non‐enzymatic process mimics the Fluc reaction itself, involving
the formation of an activated ester intermediate, followed by oxidation
(Figures S2--S3).{\protect\hyperlink{cbic201600564-bib-0032}{32}} When
the brominated analogues were subjected to the assay, light emission was
observed at levels above those for 6′‐deoxyLuc and 6′‐methoxyLuc (known
poor light emitters). More specifically, 5′‐BrLuc and 7′‐BrLuc exhibited
nearly identical levels of chemiluminescence
(Figure \protect\hyperlink{cbic201600564-fig-0003}{4}), whereas the
photon emission value for 4′‐BrLuc was slightly reduced. These results
suggested that C7′‐ and C4′‐modified luciferins are capable of
bioluminescent light emission, provided that a mutant enzyme can be
identified to accommodate them. It should also be noted that deviations
between the computed oscillator strengths
(Table \protect\hyperlink{cbic201600564-tbl-0001}{1}) and the observed
emission strengths
(Figure \protect\hyperlink{cbic201600564-fig-0003}{4}) suggested the
existence of fast, nonradiative deactivation pathways for the analogues.
Although an exhaustive study of such decay mechanisms is beyond our
present scope, additional calculations revealed dark states that are
energetically accessible from S\textsubscript{1}: 1) T\textsubscript{1}
triplet states that can mediate deactivation through intersystem
crossing, and 2) twisted intramolecular charge‐transfer (TICT) states
with near‐zero oscillator strengths due to broken pi conjugation
(Supporting Information). In chemiluminescence experiments, these
compounds would exhibit emission profiles from a range of torsional
angles, only some of which would be light emitting.

%TODO chemiluminescence
\begin{figure}
\centering
\includegraphics{//wol-prod-cdn.literatumonline.com/cms/attachment/8d487be4-0ef5-42aa-bd2f-b136ad47996a/cbic201600564-fig-0003-m.jpg}
\caption{\textbf{Figure 3} ¶ \protect\hyperlink{}{Open in figure
viewer}\href{/action/downloadFigures?id=cbic201600564-fig-0003\&doi=10.1002\%2Fcbic.201600564}{PowerPoint}
¶ Chemiluminescent light production with luciferin analogues.
A) Luciferin scaffolds were activated to form phenyl esters. Subsequent
treatment with KOPh resulted in light emission; a) PhOC(O)Cl,
{[}D\textsubscript{6}{]}DMSO, mesitylene; b) 0.1 {m} KOPh (DMSO).
B) Chemiluminescence observed with luciferin analogues.
*** \emph{p}\textless{}0.001 (t‐test). ¶
\protect\hyperlink{}{{Caption}\emph{}} ¶ Chemiluminescent light
production with luciferin analogues. A) Luciferin scaffolds were
activated to form phenyl esters. Subsequent treatment with KOPh resulted
in light emission; a) PhOC(O)Cl, {[}D6{]}DMSO, mesitylene; b) 0.1 m KOPh
(DMSO). B) Chemiluminescence observed with luciferin analogues.
*** p\textless{}0.001 (t‐test).}
\end{figure}

Our computational and experimental data suggested that 5′‐BrLuc would be
suitable for BLI in cells and tissues. Indeed, when the brominated
compound was incubated with Fluc‐expressing HEK293 human embryonic
kidney cells, robust light emission was observed
(Figure \protect\hyperlink{cbic201600564-fig-0004}{5} A). Minimal
bioluminescence was produced when cultures were treated with 4′‐BrLuc
and 7′‐BrLuc, consistent with in vitro data. Interestingly, light
emission values from cells incubated with 5′‐BrLuc were on par with
those from {d}‐Luc‐treated cultures
(Figure \protect\hyperlink{cbic201600564-fig-0004}{5}). At low doses,
5′‐BrLuc even outperformed the native substrate, suggesting that the
analogue was more cell
permeable.{\protect\hyperlink{cbic201600564-bib-0033}{33}} Similar
trends were observed in vivo. When 5′‐BrLuc was injected into mice
bearing Fluc‐expressing DB7 cells, light emission was observed
(Figure \protect\hyperlink{cbic201600564-fig-0005}{6} A). The overall
photon output was lower than that of a comparable dose of {d}‐luciferin
(Figure \protect\hyperlink{cbic201600564-fig-0005}{6} B). However, the
light emission from mice treated with 5′‐BrLuc was more sustained
(Figure \protect\hyperlink{cbic201600564-fig-0005}{6} C).

% TODO Cell data
\begin{figure}
\centering
\includegraphics{//wol-prod-cdn.literatumonline.com/cms/attachment/6c37d33c-a07b-40c4-b86c-134fbb0c1e11/cbic201600564-fig-0004-m.jpg}
\caption{\textbf{Figure 4} ¶ \protect\hyperlink{}{Open in figure
viewer}\href{/action/downloadFigures?id=cbic201600564-fig-0004\&doi=10.1002\%2Fcbic.201600564}{PowerPoint}
¶ Differential photon production observed with Fluc‐expressing cells
treated with bromo luciferins. Compounds were administered to
Fluc‐expressing HEK293 cells (100 000 cells per well) in
phosphate‐buffered saline (pH 7.4). Sample images are included above
each bar. A) Peak emission for all analogues at 100 μ{m}.
B) Dose--response comparison between 5′‐BrLuc (black) and {d}‐Luc
(gray). For A--B, error bars represent the standard deviation of the
mean for \emph{n}=3 experiments. * \emph{p}\textless{}0.1,
** \emph{p}\textless{}0.01 (t‐test). ¶
\protect\hyperlink{}{{Caption}\emph{}} ¶ Differential photon production
observed with Fluc‐expressing cells treated with bromo luciferins.
Compounds were administered to Fluc‐expressing HEK293 cells (100 000
cells per well) in phosphate‐buffered saline (pH 7.4). Sample images are
included above each bar. A) Peak emission for all analogues at 100 μm.
B) Dose--response comparison between 5′‐BrLuc (black) and d‐Luc (gray).
For A--B, error bars represent the standard deviation of the mean for
n=3 experiments. * p\textless{}0.1, ** p\textless{}0.01 (t‐test).}
\end{figure}

% TODO mouse data
\begin{figure}
\centering
\includegraphics{//wol-prod-cdn.literatumonline.com/cms/attachment/a9784712-735c-4936-b590-2f2002ed3990/cbic201600564-fig-0005-m.jpg}
\caption{\textbf{Figure 5} ¶ \protect\hyperlink{}{Open in figure
viewer}\href{/action/downloadFigures?id=cbic201600564-fig-0005\&doi=10.1002\%2Fcbic.201600564}{PowerPoint}
¶ In vivo imaging with a bromo luciferin analogue. A) Luciferins (100 μL
of 1 m{m} solutions) were administered into mice bearing Fluc‐expressing
DB7 cells. Bioluminescent images are shown. B) Quantification of the
images shown in (A). Error bars represent the standard error of the mean
(\emph{n}=3). C) Bromo luciferin analogue enables sustained imaging.
Luciferins (100 μL of 1 m{m} solutions) were administered (i.v.) into a
luciferase transgenic mouse, and bioluminescence images were acquired
over time. Data are representative of \emph{n}=3 independent
experiments. ¶ \protect\hyperlink{}{{Caption}\emph{}} ¶ In vivo imaging
with a bromo luciferin analogue. A) Luciferins (100 μL of 1 mm
solutions) were administered into mice bearing Fluc‐expressing DB7
cells. Bioluminescent images are shown. B) Quantification of the images
shown in (A). Error bars represent the standard error of the mean (n=3).
C) Bromo luciferin analogue enables sustained imaging. Luciferins
(100 μL of 1 mm solutions) were administered (i.v.) into a luciferase
transgenic mouse, and bioluminescence images were acquired over time.
Data are representative of n=3 independent experiments.}
\end{figure}

As noted previously, the brominated luciferins are useful entry points
for new probe development, as they can be further modified by using
traditional cross‐coupling reactions. As proof of principle, we used
Stille coupling conditions to install a phenyl substituent on the
luciferin core (Scheme \protect\hyperlink{cbic201600564-fig-5002}{7}).
These conditions can be used at a late stage in luciferin synthesis and
should translate well across the analogue series, enabling new families
of probes to be readily accessed. We are also exploring other
cross‐coupling methodologies to derivatize luciferins with alkyl and
other appendages, and these results will be published in due course.

% TODO cross coupling
\begin{figure}
\centering
\includegraphics{//wol-prod-cdn.literatumonline.com/cms/attachment/1bf86b5e-60af-4475-9963-d78effac3906/cbic201600564-fig-5002-m.jpg}
\caption{\textbf{Scheme 2} ¶ \protect\hyperlink{}{Open in figure
viewer}\href{/action/downloadFigures?id=cbic201600564-fig-5002\&doi=10.1002\%2Fcbic.201600564}{PowerPoint}
¶ Brominated luciferins can be derivatized by Stille cross‐coupling.
a) Pd(PPh\textsubscript{3})\textsubscript{4} (20 mol \%), LiCl
(1.2 equiv), dioxane, 120 °C, 6 h (31 \%). ¶
\protect\hyperlink{}{{Caption}\emph{}} ¶ Brominated luciferins can be
derivatized by Stille cross‐coupling. a) Pd(PPh3)4 (20 mol \%), LiCl
(1.2 equiv), dioxane, 120 °C, 6 h (31 \%).}
\end{figure}

In conclusion, we developed a series of brominated luciferins that are
versatile probes for bioluminescence. The analogues were syntheszied
from a common synthetic route, and their light‐emitting properties were
investigated by using a combination of photophysical assays. One of the
probes enabled sensitive imaging in cultured cells and animals; the
others are candidates for orthogonal probe development. Importantly,
these scaffolds can be further diversified to access next‐generation
bioluminescent tools.

\bibliographystyle{achemso}
\bibliography{thesis2}

%%% Local Variables: ***
%%% mode: latex ***
%%% TeX-master: "thesis.tex" ***
%%% End: ***
