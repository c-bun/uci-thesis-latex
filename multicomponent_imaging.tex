% !TEX root = ./thesis.tex

\chapter{Multicomponent bioluminescence imaging via substrate unmixing}
\section{Introduction}
A cornerstone of fluorescence imaging is the ability to monitor multiple subjects in tandem.\cite{RodriguezGrowingGlowingToolbox2017}
Indeed, spatiotemporal information about a species of interest is of limited use without the ability to also visualize its interacting partners. Multicolored fluorescent probes have brought a multitude of cellular interactions to light, from the interactions between immune and cancer cells\cite{GermainDecadeImagingCellular2012}, to the contacts formed between neurons.\cite{ChoiInterregionalsynapticmaps2018}
However, these studies are limited to short length scales.
Fluorescence imaging in thick tissues, or in entire opaque organisms remains a challenge. Excitation light causes autofluorescence of endogenous fluorophores (such as flavin and heme), and all but red emission light is absorbed and scattered by tissue.\cite{ZhaoEmissionspectrabioluminescent2005}

Bioluminescence imaging (BLI) is a technique complementary to fluorescence imaging that enables spatiotemporal localization of cells throughout the body of mice and other model organisms. BLI utilizes genetically encoded luciferase enzymes and exogenously administered luciferin substrates to emit photons of light. Because tissue does not normally glow, this technique is exquisitely sensitive, with the ability to detect 1-10 cells in a mouse.
This sensitivity has made BLI the go-to technique for monitoring cell proliferation and migration in mouse models.\cite{PaleyBioluminescenceversatiletechnique2014}
Despite the utility of the tool, multicomponent imaging has yet to be developed for BLI. With such a capability, longstanding questions regarding cellular interactions in an organismal context could be answered. Different classes of immune cells could be tracked as they respond to disease states, stem cells labeled with different probes could be visualized as they form organoids, or the cellular makeup early metastases could monitored.

% TODO Jenn: addl paragraph here on traditional resolution via color and substrate.
Early efforts in multicomponent BLI focused on using the natural substrate resolution that is present in nature.
Initial reports paired firefly luciferase with \it{Gaussia} or \it{Renilla} luciferases.\cite{VilaltaDualluciferaselabelling2009,WendtvivoDualSubstrate2011,BhaumikOpticalimagingRenilla2004}
These studies used the resolved probes in the same cell to simultaneously monitor cell localization and proliferation as well as expression of a gene of interest.
% TODO something here about how only had to wait two hours for imaging?
Three component imaging was first shown through imaging with firefly luciferase, \it{Gaussia} luciferase, and \it{Vargula} luciferase.\cite{MaguireTripleBioluminescenceImaging2013}
The authors used their three probes to study the effects of gene therapy on glioma proliferation. Glioma cells were monitored with firefly luciferase, gene delivery was quantified via \it{Vargula} luciferase, and \it{Gaussia} luciferase monitored NF-kB expression (a cell death marker).
However, substrates were administered on subsequent days, limiting time resolution to 72 hours, and \it{Vargula} luciferin is not widely available to the community.

Color resolution has also been pursued as a means of multicomponent BLI. Click beetle luciferases developed by Promega have been popular for such applications.\cite{MezzanotteSensitivedualcolor2011,GammonSpectralUnmixingMulticolored2006,DanielDualColorBioluminescenceImaging2015}
Though most applications of color-resolved BLI have been shown \it{in vitro},\cite{GammonSpectralUnmixingMulticolored2006} (\highlight{cite more here}) several have also demonstrated utility \it{in vivo}.\cite{MezzanotteSensitivedualcolor2011,DanielDualColorBioluminescenceImaging2015} These studies were conducted at shallow tissue depth, or with large amounts of cells to mitigate red-shifting of signal due to tissue absorption.

We have set out to develop robust, multicomponent bioluminescence imaging.
First, we developed a variety new synthetic techniques to enable diversification of firefly luciferin.\cite{McCutcheonRapidscalableassembly2015,McCutcheonExpedientsynthesiselectronically2012,SteinhardtDesignSynthesisAlkynyl2016,SteinhardtBrominatedLuciferinsAre2016,JonesOrthogonalLuciferaseLuciferinPairs2017}
Next, we mutated firefly luciferase to selectively process these luciferin analogs.\cite{JonesOrthogonalLuciferaseLuciferinPairs2017,RathbunParallelScreeningRapid2017}
These orthogonal luciferin-luciferase pairs retained selectivity in mouse models.\cite{RathbunParallelScreeningRapid2017} In this work, we demonstrate the ability of these probes to distinguish mixed populations, and apply this methodology to a cancer metastasis model.

% Figure: overall strategy %%%%%%%%%%%%%%%%%%%%%%%%%%%%%%%%%%%%%%%%%%%%%%%%%%%%%
\begin{figure}[htbp]
\includegraphics[width = \textwidth]{multicomponent_imaging/mouse_diagram}
\centering
\caption[Bioluminescent substrate unmixing for the study of disease states]{
Bioluminescent substrate unmixing for the study of disease states. Sequential substrate addition enables unmixing of bioluminescent signal. With the ability to track multiple cell types simultaneously, the dynamics of disease states can be tracked over time.
}
  \label{fig:mouse_diagram}
\end{figure}
%%%%%%%%%%%%%%%%%%%%%%%%%%%%%%%%%%%%%%%%%%%%%%%%%%%%%%%%%%%%%%%%%%%%%%%%%%%%%%%%

Because we are using substrate resolution (not color resolution) for multicomponent imaging, we could not rely on optical filters or spectral unmixing algorithms to distinguish signals. Thus, we developed a sequential imaging approach coupled with a \it{substrate} unmixing algorithm to distinguish orthogonal pairs (\autoref{fig:mouse_diagram}). We envisioned sequential substrate injection, with images acquired after each was administered. Substrate light output would ``build''  as images were acquired, and a processing step would unmix the signals to give the final relative abundance of each probe. Imaging in this way over time would enable study of disease states throughout the body of a mouse.

\section{Results and Discussion}
\subsection{Linear unmixing algorithms distinguish orthogonal bioluminescent probes}
To realize our goal, we needed to select an ideal pair of luciferin analogs and luciferase mutants that would provide reliable substrate resolution.
Bioluminescent probes for \it{in vivo} multicomponent imaging must be bright, bioavailable molecules. They must also be synthesized in large quantities. With this in mind, we examined top pairings from our parallel screens\cite{RathbunParallelScreeningRapid2017} and chose \dluc{} and \bf{4'-BrLuc} as our lead pair.
\dluciferin{} is a molecule that is familiar to the community, commercially available, and have proven robust in a vast range of imaging studies.\cite{PaleyBioluminescenceversatiletechnique2014}
\bf{4'-BrLuc} is also a well-vetted molecule that has been the subject of several studies in our lab. Its synthesis is scalable and straightforward,\cite{SteinhardtBrominatedLuciferinsAre2016} and initial \it{in vivo} studies have shown that it retains selectivity and brightness \it{in vivo.}\cite{RathbunParallelScreeningRapid2017}

\begin{thoughts}
  It is also crucial that these molecules are intensity resolved. Meaning, the absolute brightness range of each molecule shouldn't overlap with the other. In this case \bf{4'-BrLuc} gives about an order of magnitude less light than \dluc{}. Thus, \bf{4'-BrLuc} can be imaged first, followed immediately by \dluc{}. Becuase \dluc{} is an order of magnitude brighter, the \bf{4'-BrLuc} signal is entirely within the noise of the image.
\end{thoughts}
%TODO Something here about how these molecules also tend to be intensity resolved? figure?? or work in comment above??

% Figure: initial pairs TODO update this with addl muts %%%%%%%%%%%%%%%%%%%%%%%%%%%%%%%%%%%%%%%%%%%%%%%%%%%
\begin{figure}[htbp]
\includegraphics[width = 8.9cm]{multicomponent_imaging/initial_pairs}
\centering
\caption[Initial mutants of interest for multicomponent bioluminescence imaging]{Initial mutants of interest for multicomponent bioluminescence imaging. Target compound-mutant pairs were identified via parallel screening.\cite{RathbunParallelScreeningRapid2017} Each mutant selected demonstrated greater than 10-fold resolution in bacterial lysate. Compounds were administered at 100 \textmu{}M.
}
  \label{fig:initial_pairs}
\end{figure}
%%%%%%%%%%%%%%%%%%%%%%%%%%%%%%%%%%%%%%%%%%%%%%%%%%%%%%%%%%%%%%%%%%%%%%%%%%%%%%%%

Next, we set out to identify mutant luciferases that would be amenable to multicomponent imaging \it{in vivo} (\autoref{fig:mouse_diagram}). We aimed for two mutants that demonstrated greater than ten-fold selectivity with their matched luciferin, and retained robust light output. From our parallel screening data, we identified six mutants; three paired with \bf{4'-BrLuc} (50, 51, 53, and 54) and three with \dluc{} (37, 85, 86, 87, and 93) (\autoref{fig:initial_pairs} and \autoref{table:Mutants}).
Mutants 50, 51, 53, and 54 harbored similar mutations, with each changing Ser 347 to Gly or Ala. It is likely that the removal of the serine hydroxyl group opens space in the active site for the bromine.
\dluc{} was positively paired with mutants 37, 85, 86, 87, and 93; each with a mutation at Arg 218 to Lys. This mutation alone (in the case of mutant 37) is enough to drastically reduce light emission with \bf{4'-BrLuc} and has been characterized in several other studies (\highlight{cite some R218K work}).

% Table: mutants considered
\begin{table}[]
  \centering
  \caption[Target mutants employed for resolution of \bf{4'-BrLuc} and \dluc{}]{Target mutants employed for resolution of \bf{4'-BrLuc} and \dluc{}. Mutants 50, 51, 53, and 54 preferred \bf{4'-BrLuc} while 37, 85, 86, 87, and 93 preferred \dluc{}.
  }\label{table:Mutants}
  \begin{tabular}{@{}c | cccccccccc@{}}
  \toprule
  Enzyme & 218                               & 240                               & 241                               & 243                               & 247                               & 249                               & 250                               & 314                               & 316                               & 347                               \\ \midrule
  WT     & \textit{R}                        & \textit{V}                        & \textit{V}                        & \textit{F}                        & \textit{F}                        & \textit{M}                        & \textit{F}                        & \textit{S}                        & \textit{G}                        & \textit{S}                        \\
  50     & {\color[HTML]{333333} R}          & {\color[HTML]{FE0000} \textbf{I}} & {\color[HTML]{333333} V}          & {\color[HTML]{333333} F}          & {\color[HTML]{FE0000} \textbf{Y}} & {\color[HTML]{333333} M}          & {\color[HTML]{333333} F}          & {\color[HTML]{333333} S}          & {\color[HTML]{333333} G}          & {\color[HTML]{FE0000} \textbf{G}} \\
  51     & {\color[HTML]{333333} R}          & {\color[HTML]{333333} V}          & {\color[HTML]{333333} V}          & {\color[HTML]{FE0000} \textbf{M}} & {\color[HTML]{333333} F}          & {\color[HTML]{333333} M}          & {\color[HTML]{333333} F}          & {\color[HTML]{333333} S}          & {\color[HTML]{333333} G}          & {\color[HTML]{FE0000} \textbf{G}} \\
  53     & {\color[HTML]{333333} R}          & {\color[HTML]{FE0000} \textbf{I}} & {\color[HTML]{333333} V}          & {\color[HTML]{FE0000} \textbf{M}} & {\color[HTML]{FE0000} \textbf{Y}} & {\color[HTML]{333333} M}          & {\color[HTML]{333333} F}          & {\color[HTML]{333333} S}          & {\color[HTML]{333333} G}          & {\color[HTML]{FE0000} \textbf{G}} \\
  54     & {\color[HTML]{333333} R}          & {\color[HTML]{333333} V}          & {\color[HTML]{FE0000} \textbf{A}} & {\color[HTML]{333333} F}          & {\color[HTML]{FE0000} \textbf{L}} & {\color[HTML]{333333} M}          & {\color[HTML]{333333} F}          & {\color[HTML]{333333} S}          & {\color[HTML]{333333} G}          & {\color[HTML]{FE0000} \textbf{A}} \\\midrule
  37     & {\color[HTML]{FE0000} \textbf{K}} & {\color[HTML]{333333} V}          & {\color[HTML]{333333} V}          & {\color[HTML]{333333} F}          & {\color[HTML]{333333} F}          & {\color[HTML]{333333} M}          & {\color[HTML]{333333} F}          & {\color[HTML]{333333} S}          & {\color[HTML]{333333} G}          & {\color[HTML]{333333} S}          \\
  85     & {\color[HTML]{FE0000} \textbf{K}} & {\color[HTML]{333333} V}          & {\color[HTML]{333333} V}          & {\color[HTML]{333333} F}          & {\color[HTML]{333333} F}          & {\color[HTML]{333333} M}          & {\color[HTML]{FE0000} \textbf{M}} & {\color[HTML]{FE0000} \textbf{T}} & {\color[HTML]{FE0000} \textbf{T}} & {\color[HTML]{333333} S}          \\
  86     & {\color[HTML]{FE0000} \textbf{K}} & {\color[HTML]{333333} V}          & {\color[HTML]{333333} V}          & {\color[HTML]{333333} F}          & {\color[HTML]{333333} F}          & {\color[HTML]{FE0000} \textbf{L}} & {\color[HTML]{333333} F}          & {\color[HTML]{FE0000} \textbf{C}} & {\color[HTML]{333333} G}          & {\color[HTML]{333333} S}          \\
  87     & {\color[HTML]{FE0000} \textbf{K}} & {\color[HTML]{333333} V}          & {\color[HTML]{333333} V}          & {\color[HTML]{333333} F}          & {\color[HTML]{333333} F}          & {\color[HTML]{333333} M}          & {\color[HTML]{FE0000} \textbf{Y}} & {\color[HTML]{FE0000} \textbf{T}} & {\color[HTML]{FE0000} \textbf{T}} & {\color[HTML]{333333} S}          \\
  93     & {\color[HTML]{FE0000} \textbf{K}} & {\color[HTML]{333333} V}          & {\color[HTML]{333333} V}          & {\color[HTML]{333333} F}          & {\color[HTML]{333333} F}          & {\color[HTML]{FE0000} \textbf{L}} & {\color[HTML]{333333} F}          & {\color[HTML]{FE0000} \textbf{V}} & {\color[HTML]{333333} G}          & {\color[HTML]{333333} S}          \\ \bottomrule
  \end{tabular}
\end{table}
%%%%%%%

For a proof-of-concept of our technique, we designed an experiment to simulate multicomponent imaging \it{in vivo} at a variety of probe ratios (\autoref{fig:bacterial_unmixing}). \it{E. coli} expressing the mutants of interest were lysed, and lysates were mixed in a gradient of 64 concentrations (\autoref{fig:bacterial_unmixing}A). \bf{4'-BrLuc} was administered to each well, and an image of the plate was acquired. \dluciferin{} was next added to the same 64 wells, and a second image was acquired.

% Figure: bacterial unmixing %%%%%%%%%%%%%%%%%%%%%%%%%%%%%%%%%%%%%%%%%%%%%%%%%%%
\begin{figure}[htbp]
\includegraphics[width = \textwidth]{multicomponent_imaging/unmixing_intro}
\centering
\caption[Gradients of bacterial lysate test the success of substrate unmixing]{
Gradients of bacterial lysate test the success of substrate unmixing.

}
  \label{fig:unmixing_intro}
\end{figure}
%%%%%%%%%%%%%%%%%%%%%%%%%%%%%%%%%%%%%%%%%%%%%%%%%%%%%%%%%%%%%%%%%%%%%%%%%%%%%%%%

When these images were normalized and overlaid (\autoref{fig:bacterial_unmixing}C), it was clear that the corresponding signal was not reflexive of the lysate ratio (\autoref{fig:bacterial_unmixing}B, \highlight{supplementary figure?}). While channel 1 (the first image that was acquired following addition of \bf{4'-BrLuc}), showed a linear relationship across the entire range of mutant concentrations and ratios, the second channel showed a high error (R\textsuperscript{2} = 0.598).
This result makes sense because residual light from the addition of \bf{4'-BrLuc} was present in the sample, so the second image reflected a combination of both signals.

% Figure: bacterial unmixing %%%%%%%%%%%%%%%%%%%%%%%%%%%%%%%%%%%%%%%%%%%%%%%%%%%
\begin{figure}[htbp]
\includegraphics[width = 12.68cm]{multicomponent_imaging/bacterial_unmixing_v2}
\centering
\caption[Orthogonal bioluminescent probes can be distinguished in bacterial lysate]{
Orthogonal bioluminescent probes can be distinguished in bacterial lysate. Based solely on the intensity of the light output, linear unmixing algorithms can distinguish orthogonal probes. \textbf{A)} Gradients of mutant luciferases in bacterial lysate were plated in an 8x8 square. \textbf{4'-BrLuc} was first added to each of the 64 wells, and light emission was measured. Second, \dluc{} was added to each of the same 64 wells, and light emission was again measured. The images were overlaid, and substrate unmixing was applied to the raw images. \textbf{B)} Quantification of signal in (C) fit via linear regression. The shaded area represents the 95\% confidence interval of the fit. \textbf{C)} Overlaid signal from raw images in (A). \textbf{D)} Quantification of signal after substrate unmixing fit via linear regression. The shaded area represents the 95\% confidence interval of the fit. \textbf{E)} Overlaid signal after substrate unmixing.
}
  \label{fig:bacterial_unmixing}
\end{figure}
%%%%%%%%%%%%%%%%%%%%%%%%%%%%%%%%%%%%%%%%%%%%%%%%%%%%%%%%%%%%%%%%%%%%%%%%%%%%%%%%

% TODO  Maybe draw a dotted line on the plot to show what signal would be expected?\end{thoughts}


To produce an output that is more reflective of the abundance of each mutant in the mixed samples, we turned to linear unmixing, a technique typically used in the field of fluorescence imaging (\highlight{cite something here}). Linear unmixing algorithms are commonly used in fluorescence microscopy when visualizing two fluorophores that have partially overlapping spectra. Overlap in emission wavelengths can cause convolution of signal that makes quantification of relative abundance difficult. We faced a similar with our substrate resolved probes. Though we were not concerned with the color of the bioluminescent emission, we thought that these same algorithms could be used to unmix \it{substrate} overlap. Indeed, when unmixing is applied to the gradient of mixed lysate, channel two is distinguished much more accurately (\autoref{fig:bacterial_unmixing}D and E). This is apparent visually in \autoref{fig:bacterial_unmixing}E, as well as with the R\textsuperscript{2} value of channel 2 in \autoref{fig:bacterial_unmixing}D.
With this technique we were able to detect the presence of 6 \textmu{}L of mutant \bf{54} lysate in 45 \textmu{}L of mutant \bf{93} lysate (and the reverse) (\highlight{Maybe highlight these wells/points in the figure??}).

\begin{thoughts}
  Probably would be nice to include a blinded experiment where lysate mixtures are predicted? We could then quantify how close we were able to get to the actual mixtures?
\end{thoughts}

% Figure: mammalian unmixing %%%%%%%%%%%%%%%%%%%%%%%%%%%%%%%%%%%%%%%%%%%%%%%%%%%
\begin{figure}[htbp]
%\includegraphics[width = 18.3cm]{multicomponent_imaging/DB7_unmixing}
\includegraphics[width = \textwidth]{multicomponent_imaging/DB7_unmixing}
\centering
\caption[Orthogonal bioluminescent probes can be distinguished in mammalian cells]{
Three orthogonal probes can be distinguished in mammalian cells. Gradients of cells expressing firefly luciferase mutants \bf{54} and \bf{86}, and \it{Gaussia} luciferase were plated in a triangle, with 60,000 cells per well. \bf{4'-BrLuc} (500 \textmu{}M), \dluciferin{} (500 \textmu{}M), and coelenterazine (40 \textmu{}M) were added in sequence. \bf{A)} Quantification of each channel from (B) fit via linear regression. The shaded area represents the 95\% confidence interval of the fit. \bf{B)} Overlay of raw signal from mixed images. \bf{C)} Quantification of each channel from the unmixed image in (D) fit via linear regression. The shaded area represents the 95\% confidence interval of the fit. \bf{D)} Overlay of the unmixed channels.
}
  \label{fig:DB7_unmixing}
\end{figure}
%%%%%%%%%%%%%%%%%%%%%%%%%%%%%%%%%%%%%%%%%%%%%%%%%%%%%%%%%%%%%%%%%%%%%%%%%%%%%%%%

Next, we sought to test this method in mammalian cells, and determine whether we could unmix three luciferin signals (\autoref{fig:DB7_unmixing}). For these experiments we added \it{Gaussia} luciferase and coelenterazine (CTZ) as our third luciferase-luciferin pair (Channel 3), with mouse DB7 cells stably expressing mutants \bf{54}, \bf{86}, and \bf{Gluc}. Mixtures of cells totaling 60,000 were plated in a triangle of gradients and imaged between substrate additions (\bf{4'-BrLuc}, \dluc{}, and CTZ respectively). Overlay of the channels showed significant error in channels 1 and 3 (R\textsuperscript{2} values of 0.730 and 0.364 respectively, \autoref{fig:DB7_unmixing}A and B). Upon substrate unmixing, correlations improved considerably, with the proper gradient readily visualized in the overlaid image (\autoref{fig:DB7_unmixing}C and D).

This experiment highlights the utility of this technique, as even perfectly substrate resolved pairs (in the case of firefly and \it{Gaussia} luciferases) are not perfectly resolvable without applying unmixing algorithms.

\subsection{Mixed cell populations can be distinguished \textit{in vivo}.}
% TODO need a plot with quantification of mouse signals?
% QUESTION maybe don't use unmixing here? it doesn't seem to make a difference.

Encouraged by our results \it{in vitro}, we sought to test our probes \it{in vivo}. To identify the best set of probes for our mouse studies, we first screened a panel of luciferase mutants \it{in vivo}. DB7 cells stably expressing a range of 9 different mutants were implanted into the backs of mice. The next day, substrates were administered sequentially, and images were acquired between injections.

% Figure: mouse screen %%%%%%%%%%%%%%%%%%%%%%%%%%%%%%%%%%%%%%%%%%%%%%%%%%%
\begin{figure}[htbp]
\includegraphics[width = 12cm]{multicomponent_imaging/mouse_screen}
\centering
\caption[Orthogonal mutant screen in mice]{
A panel of orthogonal mutants were screened in mice to choose an optimal pair for further experimentation.
}
  \label{fig:mouse_screen}
\end{figure}
%%%%%%%%%%%%%%%%%%%%%%%%%%%%%%%%%%%%%%%%%%%%%%%%%%%%%%%%%%%%%%%%%%%%%%%%%%%%%%%%

\begin{thoughts}
  It might be nice to show here that unmixing was not needed, and introduce the concept of intensity resolution here.
\end{thoughts}

% Figure: mouse unmixing %%%%%%%%%%%%%%%%%%%%%%%%%%%%%%%%%%%%%%%%%%%%%%%%%%%
\begin{figure}[htbp]
\includegraphics[width = 14cm]{multicomponent_imaging/mouse_unmixing}
\centering
\caption[Orthogonal bioluminescent probes can be distinguished in mice]{
Orthogonal bioluminescent probes can be distinguished in mice.
Compounds were administered sequentially via i.p. injection.
}
  \label{fig:mouse_unmixing}
\end{figure}
%%%%%%%%%%%%%%%%%%%%%%%%%%%%%%%%%%%%%%%%%%%%%%%%%%%%%%%%%%%%%%%%%%%%%%%%%%%%%%%%
\bf{Points to make here:}
\begin{itemize}
  \item statements about how this is hard with tissue penetrance bioavailability etc.
  \item mouse screen because didn't know how cmpds/muts would perform
  \item mixed mouse expt with 37 and 51
\end{itemize}

\section{Conclusions and future directions}

\bibliographystyle{achemso}
\bibliography{bibs/zoteroLib}

%%% Local Variables: ***
%%% mode: latex ***
%%% TeX-master: "thesis.tex" ***
%%% End: ***
