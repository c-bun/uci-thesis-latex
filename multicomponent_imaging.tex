\chapter{Multicomponent bioluminescence imaging via substrate unmixing}
\section{Introduction}

% Figure: overall strategy %%%%%%%%%%%%%%%%%%%%%%%%%%%%%%%%%%%%%%%%%%%%%%%%%%%%%
\begin{figure}[htb]
\includegraphics[width = \textwidth]{multicomponent_imaging/mouse_diagram}
\centering
\caption[Bioluminescent substrate unmixing for the study of disease states]{
Bioluminescent substrate unmixing for the study of disease states. Sequential substrate addition enables unmixing of bioluminescent signal. With the ability to track multiple cell types simultaneously, the dynamics of disease states can be tracked over time.
}
  \label{fig:mouse_diagram}
\end{figure}
%%%%%%%%%%%%%%%%%%%%%%%%%%%%%%%%%%%%%%%%%%%%%%%%%%%%%%%%%%%%%%%%%%%%%%%%%%%%%%%%


\section{Results and Discussion}
%%\subsection{Bioluminescent probes give reproducible patterns of substrate usage}

\subsection{Linear unmixing algorithms distinguish orthogonal bioluminescent probes}
% TODO need a figure here with the linear plots as well.
% Figure: bacterial unmixing %%%%%%%%%%%%%%%%%%%%%%%%%%%%%%%%%%%%%%%%%%%%%%%%%%%
\begin{figure}[htb]
\includegraphics[width = \textwidth]{multicomponent_imaging/bacterial_unmixing}
\centering
\caption[Orthogonal bioluminescent probes can be distinguished in bacterial lysate]{
Orthogonal bioluminescent probes can be distinguished in bacterial lysate. Based solely on the intensity of the light output, linear unmixing algorithms can distinguish orthogonal probes. Between images, substrates were added to all wells.
}
  \label{fig:bacterial_unmixing}
\end{figure}
%%%%%%%%%%%%%%%%%%%%%%%%%%%%%%%%%%%%%%%%%%%%%%%%%%%%%%%%%%%%%%%%%%%%%%%%%%%%%%%%

\subsection{Mixed cell populations can be distinguished \textit{in vivo}.}
% TODO need a plot with quantification of mouse signals?
% Figure: mouse unmixing %%%%%%%%%%%%%%%%%%%%%%%%%%%%%%%%%%%%%%%%%%%%%%%%%%%
\begin{figure}[htb]
\includegraphics[width = \textwidth]{multicomponent_imaging/mouse_unmixing}
\centering
\caption[Orthogonal bioluminescent probes can be distinguished in mice]{
Orthogonal bioluminescent probes can be distinguished in mice.
Compounds were administered sequentially via i.p\. injection.
}
  \label{fig:mouse_unmixing}
\end{figure}
%%%%%%%%%%%%%%%%%%%%%%%%%%%%%%%%%%%%%%%%%%%%%%%%%%%%%%%%%%%%%%%%%%%%%%%%%%%%%%%%

\section{Conclusions and future directions}

\bibliographystyle{achemso}
\bibliography{thesis2}

%%% Local Variables: ***
%%% mode: latex ***
%%% TeX-master: "thesis.tex" ***
%%% End: ***
